%***************************************************************************
% MCLab Protocol Template
%
% Embedded Computing Systems Group
% Institute of Computer Engineering
% TU Vienna
%
%---------------------------------------------------------------------------
% Vers. Author Date     Changes
% 1.0   bw     10.03.06 first version
% 1.1   bw     25.04.06 listing is in a different directory
% 1.2   bw     24.05.06 tutor has to be listed on title page
% 1.3   bw     16.06.06 statement about no plagiarism on title page (sign it!)
% 1.4   mp     21.07.15 changed coversheet, rewording
%---------------------------------------------------------------------------
% Author names:
%       bw      Bettina Weiss
%       mp      Martin Perner
%***************************************************************************

\documentclass[12pt,a4paper,titlepage,oneside]{article}
\usepackage{graphicx}            % fuer Bilder
\usepackage{listings}            % fuer Programmlistings
%\usepackage{german}              % fuer deutsche Umbrueche
\usepackage[latin1]{inputenc}    % fuer Umlaute
\usepackage{times}               % PDF files look good on screen
\usepackage{amssymb,amsmath,amsthm}
\usepackage{url}
\usepackage{enumitem}
\usepackage{fullpage}
\usepackage{tikz}
\usepackage[capitalize,nameinlink,noabbrev]{cleveref}

\lstset{language=C,
	frame=single,
	captionpos=b,
}


%***************************************************************************
% note: the template is in English, but you can use German for your
% protocol as well; in that case, remove the comment from the
% \usepackage{german} line above
%***************************************************************************


%***************************************************************************
% enter your data into the following fields
%***************************************************************************
\newcommand{\Vorname}{Jan}
\newcommand{\Nachname}{Nausner}
\newcommand{\MatrNr}{01614835}
\newcommand{\Email}{e01614835@student.tuwien.ac.at}
\newcommand{\Part}{I}
%***************************************************************************


%***************************************************************************
% generating the document from Protocol.tex:
%       "pdflatex Protocol"        generates a .pdf file
%       "pdflatex Protocol"        repeat to get correct table of contents
%       "evince Protocol.pdf &"    shows the .pdf file on viewer
%
%***************************************************************************

%---------------------------------------------------------------------------
% include all the stuff that is the same for all protocols and students
\input ProtocolHeader.tex
%---------------------------------------------------------------------------

\begin{document}

%---------------------------------------------------------------------------
% create titlepage and table of contents
\MakeTitleAndTOC
%---------------------------------------------------------------------------


%***************************************************************************
% This is where your protocol starts
%***************************************************************************

%***************************************************************************
% remove the following lines from your own protocol file!
%***************************************************************************
%\noindent
%\textbf{Note:} This template is provided to show you how \LaTeX{} works and may
%not contain all subsections your protocol should contain.\\
%\textbf{Note:} You can, and in fact should, reuse appropriate parts
%from the implementation proposal in the protocol.


%***************************************************************************
\section{Overview}
%***************************************************************************

%---------------------------------------------------------------------------
\subsection{Connections,  External Pullups/Pulldowns}
%---------------------------------------------------------------------------

\bConnections{What}{}
J12 & Connected to VCC \\
J14 & EXT \\
J15 & PF0 \\
LEDs & Off \\
GLCD backlight & On \\
PORT switches & Off \\
PORT jumpers & Pull down \\ 
\eConnections

The mp3-SD module needs to be connected as per specification and the bluetooth
module needs to be connected to PORTJ.

%---------------------------------------------------------------------------
\subsection{Design Decisions}
%---------------------------------------------------------------------------

For abstraction and compartementalization purposes, the application is split
up into many independant submodules, as proposed in the specification. Furthermore 
it was decided to implement a standalone timer module, which abstracts the timer
hardware and allows to start timers by just specifiying the period in ms.
Also the main application was not put in the main file, but rather in a 
seperate game module, connecting all the necessary driver modules, handling
the game logic and task switching. This leads to a clean main.c file, without
many includes or ISRs and just two function calls, one for setup and one for
running the main application loop. Another major design decision was to 
precompute a number of random walls with a python script (randomWalls.py) for the game and store
them in the flash to later load them randomly in order to trade storage for 
computation time.

%---------------------------------------------------------------------------
\subsection{Specialities}
%---------------------------------------------------------------------------

The application is very reactive to user input and runs smoothly. Also the
user interface controls can be considered as intuitive or are appropriately
explained with text. By partitioning the big background task into many subtasks
it is prevented that one task monopolizes the CPU, leading to clear music 
playback, responsive controls and smooth gameplay. Also on the screen showing
an ongoing connection attempt between MCU and wiimote an animation was included
to signifiy the user that the application is still progressing (this could be a place to award bonus points). The former
mentioned timer library can also be listed as a speciality, as it handles the
hardware in an intuitive way for the programmer and also checks if hardware
is not allocated multiple times (another place for potentially awarding bonus points).

The gameplay itself can be considered a bit boring, as it might be a littlebit too
easy for some people's taste. This could be improved by integrating harder
platforms (less or smaller gaps) and supporting even higher scroll speeds.
But this would be the task of a game UX specialist rather than that of a 
computer engineer to be.

%***************************************************************************
\section{Main Application}
%***************************************************************************

As described above, the code for the main application is not located in the
main.c file but in a seperate game.c file. When the application is started,
music starts playing,
the program checks for a present bluetooth connection and if the remote is
disconnted it initates a new connection. Provided the correct MAC-address
was entered in the mac.h file, after several secondes the program switches
to the start screen, implying a successful connection to the wiimote. On the
start screen the user is presented with two options: either start playing the
game or view the highscore table. If the highscore table is selected, it will
be empty as no persistent storage was implemented.

When the "Play" option is
selected, the user is presented a screen, where a player can be selected, using
the up and down arrows on the wiimote. At this point, the user can choose to
go back to the start screen or to start playing the game.

Upon entering the
game, the screen is filled with random walls, the ball is placed on the bottom
and the accelerometer in the wiimote is enabled in order to allow steering the
ball. The level immediately starts scrolling and placing new random walls. The
ball can be moved in x direction by holding the wiimote in an orientation similar
to how one would hold a TV remote and by tilting it right or left. Gameplay is
as described in the specification, the ball cannot move out of the visible 
field, falls down if not blocked by platforms or the bottom and the game is lost
if the ball touches the top. The game either ends when the game-over condition is
met or when the home button is pressed, which leads back to the start screen.
In both cases the current score is placed in the highscore table if it is 
sufficiently high and the accelerometer is disabled again. In case the player 
was defeated, a game-over screen with the achieved score is shown. The user
can chose to go back to the start screen or to directly view the highscore
table, sorted in descending order.

The main application is driven by a gametick timer which fires every 30 ms.
On every gametick, the user input from the wiimote is checked in order to
determine if the app needs to move to another screen. In the game a few
additional tasks need to be executed. First, the screen is updated by moving
the ball according to the user input after performing collision detection and
checking if the game has been lost. By using timer divisor counters, the screen
scrolls after several gameticks and after shifting out a wall, a new random
wall is drawn on the screen. The score counter and the difficulty (scrolling
speed) are also updated by using timer divisor counters.

%***************************************************************************
\section{Music Playback}
%***************************************************************************

%---------------------------------------------------------------------------
\subsection{SPI}
%---------------------------------------------------------------------------

The SPI module was implemented mostly by following the examples given on
pages 198-199 of the ATmega1280 manual. The module uses busy-waiting rather
than interrupts and the SPI is configured to be in master mode.

%---------------------------------------------------------------------------
\subsection{Playback}
%---------------------------------------------------------------------------

Music playback is implemented in music.c by reading a 32 bit block of data from the SD 
card and sending it to the mp3 module. The function immediately returns, so
that the game task can proceed in the meanwhile, but is called repeatedly
until the buffer of th mp3 module is full. After the end of the selected song
has been reached, it starts playing all over again. The music module also 
provides a function to set the volume on the mp3 board, which linearizes the
input value and can be used as a callback for the ADC module. Note that the
volume is only set in case the value has actually changed and if the SPI is not
used by the music playback task.

%***************************************************************************
\section{ADC}
%***************************************************************************

The ADC is required to fullfill two tasks: reading the the onboard
potentiometer to set the music volume and providing entropy for the PRNG
module. This is achived by setting the ADC to auto trigger mode, triggering
it every ~200 Hz, and alternatingly reading the potentiometer and the floating
pins, which produces an update of the respective values approximately every
10 ms. When the trigger timer interrupt arrives the ADC is enabled (required
if differential mode is used in combination with auto triggering according
to the manual). After finishing the conversion (ADC interrupt) the result
is read, the ADC disabled, the ADC MUX gets set to read from the next source
and the ADC result is passed to the appropriate callback.

%***************************************************************************
\section{GLCD--Display}
%***************************************************************************

%---------------------------------------------------------------------------
\subsection{GLCD}
%---------------------------------------------------------------------------

The GLCD module provides high-level functions for drawing and writing on the
display. It relies on functions for setting, clearing and inverting single
pixels. This functions need to first read the respective memory page before performing
a pixel operation, as the state of the other pixels in the page must be preserved.
The functions for drawing lines and circles implement the Bresenham Line Algorithm (https://en.wikipedia.org/wiki/Bresenham\%27s\_line\_algorithm)
and the Midpoint Circle Algorithm 
(https://en.wikipedia.org/wiki/Midpoint\_circle\_algorithm). Empty and filled
rectangles are composed of single lines, where several cases must be considered
so that no pixel gets drawn twice in order for the invert operation to
function correctly. In order to draw characters, it is required to set a pointer
to the correct entry in the font array and then draw the respective pixels
forming the character. The text writing functions call the character
drawing function for every character in a string and discard any symbols not
fitting on a given line.

%---------------------------------------------------------------------------
\subsection{HAL GLCD}
%---------------------------------------------------------------------------

The HAL GLCD module handles communication with the GLCD controller and implements
four basic operations: reading and writing data, sending commands and reading
the status bits. Before carrying out the first three cases, it must be ensured
that the GLCD is reset and not busy, which is ensured by busy waiting until
the corresponding status bits are set. After writing data or commands and 
before reading data, the respective control bits are set, the correct controller is selected and the enable flag is
toggled as per timing diagram in the datasheet, which is realized by waiting
with a few NOPs. The high level read, write and address-setting functions
abstract away the fact that the GLCD in fact uses two seperate controllers
by maintaining an internal state which controller needs to be used. This
state is updated upon crossing memory boundaries. The y-shift is set with
a dedicated command and the current value is tracked by an internal variable.
The read function needs to perform two read operations where the first result
is discarded, as the controllers require a dummy read.

%***************************************************************************
\section{PRNG}
%***************************************************************************

The pseudorandom number generator implements the linear feedback shift register
algorithm suggested by the PRNG manual using volatile inline assembler. This
assembler block needs to be atomic, as it makes heavy use of the carry bit 
and requires 16 bit arithmetic. It provides functions for feeding entropy and 
for extracting pseudorandom numbers from the LFSR.

%***************************************************************************
\section{Timer abstraction}
%***************************************************************************

The timer moduel provides abstraction for timers 1, 3, 4 and 5. The user can
specify how long the timer should count in milliseconds and if the timer should
only be fired once or if it should run periodically. The output compare registers
and prescaler values are set accordingly. The module also offers the option to
stop a periodic timer, which can for example be used for implementing animations. 
The module also provides feedback if the requested timer is available or already in use.

%***************************************************************************
\section{UART}
%***************************************************************************

The UART's baudrate is set to 1 M using double transmission speed. If a new
byte is received, the RX ISR puts it in the ringbuffer and calls the
processRingbuffer procedure (only once, ensured by lock flags), which empties
the buffer by calling the receive callback in the background. If the buffer fill level reaches a certain
limit, the CTS line is set to signify the bluetooth module that transmission
should be halted. If a byte should be sent, there are three cases which need
to be checked before sending. First, reset must be completed. If this is not
the case, the byte to be sent gets buffered and sending is tried again upon
reset completion. Second, it must be assured that the RTS hardware flow control
flag is not set by the blutooth module. Otherwise, sending is postponed again
and a pin change interrupt gets activated on the RTS pin. If the corresponding ISR
is executed, sending of the buffered byte is tried again and the PCINT disabled.
Finally, the UART data register must be empty. If it is not, the UART data register
interrupt is enabled and upon triggering, sending is tried again. If nothing
blocks the send procedure, the byte is sent and the UDRINT enabled. On
execution of the corresponding ISR the send callback is called.

%***************************************************************************
\section{Problems}
%***************************************************************************

One problem that was encountered during the development of the application was the faulty
function wiiUserSetAccel. In the specification it was stated that using this function it would be
possible to either turn the wiimote accelerometer on or off. Unfortunately this behavior was
not reflected in the implementation, which only turned the accelerometer on, regardless of the
value provided to the enable parameter.

Another difficulty concerned the interpretation of the accelerometer data. It took some
time and thinking to derive a simple and fast algorithm for wiimote tilt detection which
did not require the use of complex numerics or even floating point arithmetic. To un-
derstand the principal behavior of an accelerometer, the following article was consulted:
http://bildr.org/2011/04/sensing-orientation-with-the-adxl335-arduino/.

A lot of time had to be in invested in figuring out how to communicate with the GLCD
module. The exact timing and pin level sequence had to be found out by trial-and-error, as the
datasheets for the hardware were not completely accurate.

Finally, just before uploading the solution a bug was discovered, where the program started to
freeze or reset the board after running a considerably long time. After many hours of debugging
the error was found to be hidden in a for loop counting from high to low numbers. The problem
was that one part of the loop condition was checking for the index to be greater or equal to
zero and with the index being an unsigned integer, this sometimes resulted in an endless loop.

%***************************************************************************
\section{Work}
%***************************************************************************

\begin{tabular}{|l|c|c|}\hline
	Task & Assumption (IP) & Reality \\ \hline

	reading manuals, datasheets &  5 h &  5 h\\
	program design              & 10 h &  6 h\\
	programming                 & 20 h & 16 h\\
	debugging                   & 20 h & 60 h\\
	protocol                    & 5  h &  4 h\\ \hline

	\textbf{Total}              & 60 h & 91 h\\ \hline
\end{tabular}


%***************************************************************************
\newpage
\end{document}

