%***************************************************************************
% MCLab Protocol Template
%
% Embedded Computing Systems Group
% Institute of Computer Engineering
% TU Vienna
%
%---------------------------------------------------------------------------
% Vers. Author Date     Changes
% 1.0   bw     10.03.06 first version
% 1.1   bw     25.04.06 listing is in a different directory
% 1.2   bw     24.05.06 tutor has to be listed on title page
% 1.3   bw     16.06.06 statement about no plagiarism on title page (sign it!)
% 1.4   mp     21.07.15 changed coversheet, rewording
%---------------------------------------------------------------------------
% Author names:
%       bw      Bettina Weiss
%       mp      Martin Perner
%***************************************************************************

\documentclass[12pt,a4paper,titlepage,oneside]{article}
\usepackage{graphicx}            % fuer Bilder
\usepackage{listings}            % fuer Programmlistings
%\usepackage{german}              % fuer deutsche Umbrueche
\usepackage[latin1]{inputenc}    % fuer Umlaute
\usepackage{times}               % PDF files look good on screen
\usepackage{amssymb,amsmath,amsthm}
\usepackage{url}
\usepackage{enumitem}
\usepackage{fullpage}
\usepackage{tikz}
\usepackage[capitalize,nameinlink,noabbrev]{cleveref}

\lstset{language=C,
	frame=single,
	captionpos=b,
}


%***************************************************************************
% note: the template is in English, but you can use German for your
% protocol as well; in that case, remove the comment from the
% \usepackage{german} line above
%***************************************************************************


%***************************************************************************
% enter your data into the following fields
%***************************************************************************
\newcommand{\Vorname}{IhrVorname/Your First Name}
\newcommand{\Nachname}{IhrNachname/YourSurname}
\newcommand{\MatrNr}{3333333}
\newcommand{\Email}{e3333333@student.tuwien.ac.at}
\newcommand{\Part}{I}
%***************************************************************************


%***************************************************************************
% generating the document from Protocol.tex:
%       "pdflatex Protocol"        generates a .pdf file
%       "pdflatex Protocol"        repeat to get correct table of contents
%       "evince Protocol.pdf &"    shows the .pdf file on viewer
%
%***************************************************************************

%---------------------------------------------------------------------------
% include all the stuff that is the same for all protocols and students
\input ProtocolHeader.tex
%---------------------------------------------------------------------------

\begin{document}

%---------------------------------------------------------------------------
% create titlepage and table of contents
\MakeTitleAndTOC
%---------------------------------------------------------------------------


%***************************************************************************
% This is where your protocol starts
%***************************************************************************

%***************************************************************************
% remove the following lines from your own protocol file!
%***************************************************************************
\noindent
\textbf{Note:} This template is provided to show you how \LaTeX{} works and may
not contain all subsections your protocol should contain.\\
\textbf{Note:} You can, and in fact should, reuse appropriate parts
from the implementation proposal in the protocol.


%***************************************************************************
\section{Overview}
%***************************************************************************

%---------------------------------------------------------------------------
\subsection{Connections,  External Pullups/Pulldowns}
%---------------------------------------------------------------------------

\bConnections{What}{}
J12 & Connected to VCC \\
\eConnections

Write down all things we need to know to get your program running on our board.
All non-standard external connections, all switches your program needs, \dots
If we cannot figure out how we get your program running, we can not give you
	points for it.


%---------------------------------------------------------------------------
\subsection{Design Decisions}
%---------------------------------------------------------------------------

Here comes the design decisions that you made during programming.

%---------------------------------------------------------------------------
\subsection{Specialities}
%---------------------------------------------------------------------------

Does you solution have something special (positive or negative)?


%***************************************************************************
\section{Main Application}
%***************************************************************************

Describe your application.





%***************************************************************************
\section{Music Playback}
%***************************************************************************

%---------------------------------------------------------------------------
\subsection{SPI}
%---------------------------------------------------------------------------

Explain your modules.

%---------------------------------------------------------------------------
\subsection{Playback}
%---------------------------------------------------------------------------


%***************************************************************************
\section{LC--Display}
%***************************************************************************

%---------------------------------------------------------------------------
\subsection{GLCD}
%---------------------------------------------------------------------------

Explain your modules.

%---------------------------------------------------------------------------
\subsection{HAL GLCD}
%---------------------------------------------------------------------------



%***************************************************************************
\section{...explain your application modules ...}
%***************************************************************************

%***************************************************************************
\section{...the above were only examples}
%***************************************************************************



%***************************************************************************
\section{Problems}
%***************************************************************************

In this section you can write about the problems you encoutered while
	implementation your application.
This can range from misunderstanding of the hardware, to plain, simple, typos
	in your program.

This is important information for us, which allows us to determine if there
	were common problems and act accordingly.
Don't worry, we don't deduct points for problems mention here which are
	already fixed.


%***************************************************************************
\section{Work}
%***************************************************************************

Estimate the work you put into solving the Application.
You can add additional points, if you like.

\begin{tabular}{|l|c|c|}\hline
	Task & Assumption (IP) & Reality \\ \hline

	reading manuals, datasheets &  5 h &  5 h\\
	program design              &  5 h &  5 h\\
	programming                 & 10 h & 10 h\\
	debugging                   & 45 h & 50 h\\
	questions, protocol         & 10 h &  5 h\\ \hline

	\textbf{Total}              & 75 h & 75 h\\ \hline
\end{tabular}


%***************************************************************************
\newpage
\end{document}

