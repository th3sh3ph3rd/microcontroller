%***************************************************************************
% MCLab Protocol Template
%
% Embedded Computing Systems Group
% Institute of Computer Engineering
% TU Vienna
%
%---------------------------------------------------------------------------
% Vers. Author Date     Changes
% 1.0   bw     10.03.06 first version
% 1.1   bw     25.04.06 listing is in a different directory
% 1.2   bw     24.05.06 tutor has to be listed on title page
% 1.3   bw     16.06.06 statement about no plagiarism on title page (sign it!)
% 1.4   mp     21.07.15 changed coversheet, rewording
%---------------------------------------------------------------------------
% Author names:
%       bw      Bettina Weiss
%       mp      Martin Perner
%***************************************************************************

\documentclass[12pt,a4paper,titlepage,oneside]{article}
\usepackage{graphicx}            % fuer Bilder
\usepackage{listings}            % fuer Programmlistings
%\usepackage{german}              % fuer deutsche Umbrueche
\usepackage[latin1]{inputenc}    % fuer Umlaute
\usepackage{times}               % PDF files look good on screen
\usepackage{amssymb,amsmath,amsthm}
\usepackage{url}
\usepackage{enumitem}
\usepackage{fullpage}
\usepackage{tikz}
\usepackage[capitalize,nameinlink,noabbrev]{cleveref}

\lstset{language=C,
	frame=single,
	captionpos=b,
}


%***************************************************************************
% note: the template is in English, but you can use German for your
% protocol as well; in that case, remove the comment from the
% \usepackage{german} line above
%***************************************************************************


%***************************************************************************
% enter your data into the following fields
%***************************************************************************
\newcommand{\Vorname}{Jan}
\newcommand{\Nachname}{Nausner}
\newcommand{\MatrNr}{01614835}
\newcommand{\Email}{e01614835@student.tuwien.ac.at}
\newcommand{\Part}{I}
%***************************************************************************


%***************************************************************************
% generating the document from Protocol.tex:
%       "pdflatex Protocol"        generates a .pdf file
%       "pdflatex Protocol"        repeat to get correct table of contents
%       "evince Protocol.pdf &"    shows the .pdf file on viewer
%
%***************************************************************************

%---------------------------------------------------------------------------
% include all the stuff that is the same for all protocols and students
\input ProtocolHeader.tex
%---------------------------------------------------------------------------

\begin{document}

%---------------------------------------------------------------------------
% create titlepage and table of contents
\MakeTitleAndTOC
%---------------------------------------------------------------------------


%***************************************************************************
% This is where your protocol starts
%***************************************************************************

%***************************************************************************
% remove the following lines from your own protocol file!
%***************************************************************************
\noindent
\textbf{Note:} This template is provided to show you how \LaTeX{} works and may
not contain all subsections your protocol should contain.\\
\textbf{Note:} You can, and in fact should, reuse appropriate parts
from the implementation proposal in the protocol.


%***************************************************************************
\section{Overview}
%***************************************************************************

%---------------------------------------------------------------------------
\subsection{Connections,  External Pullups/Pulldowns}
%---------------------------------------------------------------------------

\bConnections{What}{}
J12 & Connected to VCC \\
J14 & EXT \\
J15 & PF0 \\
LEDs & Off \\
GLCD backlight & On \\
PORT switches & Off \\
PORT jumpers & Pull down 
\eConnections

Write down all things we need to know to get your program running on our board.
All non-standard external connections, all switches your program needs, \dots
If we cannot figure out how we get your program running, we can not give you
	points for it.


%---------------------------------------------------------------------------
\subsection{Design Decisions}
%---------------------------------------------------------------------------

Here comes the design decisions that you made during programming.

%---------------------------------------------------------------------------
\subsection{Specialities}
%---------------------------------------------------------------------------

Does you solution have something special (positive or negative)?


%***************************************************************************
\section{Main Application}
%***************************************************************************

Describe your application.





%***************************************************************************
\section{Music Playback}
%***************************************************************************

%---------------------------------------------------------------------------
\subsection{SPI}
%---------------------------------------------------------------------------

Explain your modules.

%---------------------------------------------------------------------------
\subsection{Playback}
%---------------------------------------------------------------------------


%***************************************************************************
\section{LC--Display}
%***************************************************************************

%---------------------------------------------------------------------------
\subsection{GLCD}
%---------------------------------------------------------------------------

Explain your modules.

%---------------------------------------------------------------------------
\subsection{HAL GLCD}
%---------------------------------------------------------------------------



%***************************************************************************
\section{...explain your application modules ...}
%***************************************************************************

%***************************************************************************
\section{...the above were only examples}
%***************************************************************************



%***************************************************************************
\section{Problems}
%***************************************************************************

One problem that was encountered during the development of the application was the faulty
function wiiUserSetAccel. In the specification it was stated that using this function it would be
4possible to either turn the wiimote accelerometer on or off. Unfortunately this behavior was
not reflected in the implementation, which only turned the accelerometer on, regardless of the
value of the enable parameter.

Another difficulty concerned the interpretation of the accelerometer data. It took some
time and thinking to derive a simple and fast algorithm for wiimote tilt detection which
did not require the use of complex numerics or even floating point arithmetic. To un-
derstand the principal behavior of an accelerometer, the following article was consulted:
http://bildr.org/2011/04/sensing-orientation-with-the-adxl335-arduino/.

A lot of time had to be in invested in figuring out how to communicate with the GLCD
module. The exact timing and pin level sequence had to be found out by trial-and-error, as the
datasheets for the hardware were not completely accurate.

Finally, just before uploading the solution a bug was discovered, where the program started to
freeze or reset the board after running a considerably long time. After many hours of debugging
the error was found to be hidden in a for loop counting from high to low numbers. The problem
was that one part of the loop condition was checking for the index to be greater or equal to
zero and with the index being an unsigned integer, this sometimes resulted in an endless loop.

%***************************************************************************
\section{Work}
%***************************************************************************

Estimate the work you put into solving the Application.
You can add additional points, if you like.

\begin{tabular}{|l|c|c|}\hline
	Task & Assumption (IP) & Reality \\ \hline

	reading manuals, datasheets &  5 h &  5 h\\
	program design              & 10 h &  6 h\\
	programming                 & 20 h & 16 h\\
	debugging                   & 20 h & 60 h\\
	protocol                    & 5  h &  4 h\\ \hline

	\textbf{Total}              & 60 h & 91 h\\ \hline
\end{tabular}


%***************************************************************************
\newpage
\end{document}

