%***************************************************************************
% MCLab Protocol Template
%
% Embedded Computing Systems Group
% Institute of Computer Engineering
% TU Vienna
%
%---------------------------------------------------------------------------
% Vers.	Author	Date	Changes
% 1.0	bw	10.3.06	first version
% 1.1	bw	25.4.06	listing is in a different directory
% 1.2	bw	24.5.06	tutor has to be listed on title page
% 1.3	bw	16.6.06	statement about no plagiarism on title page (sign it!)
%---------------------------------------------------------------------------
% Author names:
%       bw      Bettina Weiss
%***************************************************************************

\documentclass[12pt,a4paper,titlepage,oneside]{article}
\usepackage{graphicx}            % fuer Bilder
\usepackage{listings}            % fuer Programmlistings
%\usepackage{german}              % fuer deutsche Umbrueche
\usepackage[latin1]{inputenc}    % fuer Umlaute
\usepackage{times}               % PDF files look good on screen
\usepackage{amssymb,amsmath,amsthm}
\usepackage{url}
\usepackage{enumitem}
\usepackage{fullpage}
\usepackage{style}

\usepackage{bm}

\usepackage{algorithm2e}

\usepackage{pmboxdraw}
\usepackage{verbatim}
\usepackage{multirow}

\usepackage{textcomp}
\usepackage{siunitx}

\sisetup{
  per-mode=symbol,
  binary-units=true
}
\usepackage{booktabs}

\usepackage[usenames,dvipsnames]{xcolor}
\usepackage[normalem]{ulem}

\usepackage{fullpage}

\usepackage{longtable}

\definecolor{dkgreen}{rgb}{0,0.6,0}
\definecolor{gray}{rgb}{0.5,0.5,0.5}
\definecolor{mauve}{rgb}{0.58,0,0.82}

\usepackage{listings}

\lstset{language=C,
        frame=single,
        keywordstyle=\color{blue},
        commentstyle=\color{dkgreen},
        stringstyle=\color{mauve},
        basicstyle=\footnotesize,
        captionpos=b,
        morekeywords={*,command, event, interface},
        tabsize=4,
        columns=fullflexible
}

\usepackage{rotating}

\usepackage{tikz}
\usetikzlibrary{calc,shapes,arrows,decorations.pathreplacing, intersections}

\usepackage[tikz]{bclogo}

\newcommand{\pts}[2]{\textbf{[#1]} #2}

\renewcommand{\ge}[0]{\geqslant}
\renewcommand{\le}[0]{\leqslant}
\renewcommand{\geq}[0]{\geqslant}
\renewcommand{\leq}[0]{\leqslant}

%***************************************************************************
% note: the template is in English, but you can use German for your
% protocol as well; in that case, remove the comment from the
% \usepackage{german} line above
%***************************************************************************


%***************************************************************************
% enter your data into the following fields
%***************************************************************************
\newcommand{\Vorname}{Jan}
\newcommand{\Nachname}{Nausner}
\newcommand{\MatrNr}{01614835   }
\newcommand{\Email}{jan.nausner@gmail.com   }
\newcommand{\Part}{II}
\newcommand{\Tutor}{}
%***************************************************************************


%---------------------------------------------------------------------------
% include all the stuff that is the same for all protocols and students
\input ProtocolHeader.tex
%---------------------------------------------------------------------------

\begin{document}

%---------------------------------------------------------------------------
% create titlepage and table of contents
\MakeTitleAndTOC
%---------------------------------------------------------------------------


%***************************************************************************
% This is where your protocol starts
%***************************************************************************

%***************************************************************************
% remove the following lines from your own protocol file!
%***************************************************************************

%***************************************************************************
\section{Overview}
%***************************************************************************

%---------------------------------------------------------------------------
\subsection{Connections,  External Pullups/Pulldowns}
%---------------------------------------------------------------------------

\bConnections{What}{}
J12 & Connected to VCC \\
J14 & EXT \\
J15 & PF0 \\
LEDs & Off \\
GLCD backlight & On \\
PORT switches & Off \\
PORT jumpers & Pull down \\ 
\eConnections

The FMClick, Ethernet and PS2 modules need to be connected as per specification.

%---------------------------------------------------------------------------
\subsection{Design Decisions}
%---------------------------------------------------------------------------

On startup, the application tries to fetch the channel list from the desktop
PC. If it is empty, it automatically performs a band seek and syncs the so obtained
channels with the database. On band seek, the channel gets saved as soon as RDS
information is available or after a specific timeout, in order to avoid hanging
on invalid/weak radio stations. If no RDS information was available at this point,
an according meessage is stored in the note. Further design decisions are explained in the
respective module sections.

%---------------------------------------------------------------------------
\subsection{Specialities}
%---------------------------------------------------------------------------

The extended UDP functionality was implemented, which might be a source for
some bonus points.
Additionally, error handlers for soft and hard
errors were implemented in the main application. Soft errors are displayed
for a short time and the app then returns to an error free state, while hard
errors stop the app after the message was displayed.

%***************************************************************************
\section{Main Application}
%***************************************************************************

Operation of the main application centers around a channel list datastructure,
which contains the respective channelInfo struct and buffers for station name
and a note. This list gets filled on reset either by syncing with the database
or by performing a band seek operation. Afterwards, the app tunes to the first
channel in the list, displays channel information and awaits user input. The app can be controlled by the 
PS2 keyboard and with the volume potentiometer. To initiate a seek up/down
operation, '.' or ',' have to be pressed. If 't' is pressed, a frequency can be
entered (in hundreds of kHz) and after pressing enter, the app tunes to this
frequency. By pressing 's', a band seek operation is performed, which purges
the local list as well as the database and refills it with stations it found 
by seeking the band. If 'a' is pressed, the current channel is added to the list and
database or updated with new RDS information if already in the list. Pressing
'f' adds the current channel as a favourite, if already added to the list.
It is assigned the next free favourite index. A note can be entered by pressing 'n'.
It will be stored when enter was pressed. The local list can be stepped through
by pressing 'l', which tunes to the next highest channel in the list. After the 
execution of these commands, the app returns to the main screen, where information
on the current radio station is shown, If the current channel is in the list, 
c[id] is shown in the first line and if it is in the favourites, f[quick dial]
is displayed.

%***************************************************************************
\section{Communication}
%***************************************************************************

%---------------------------------------------------------------------------
\subsection{UDP}
%---------------------------------------------------------------------------

The provided UDP network stack is lacking some important features, which had
to be added manually. In order to support ICMP echo request messages, PingP
had to be extended to reflect such requests back to the source, combined with
the ICMP echo reply header. To provide an appropriate response to communication
requests on non-serviced ports, UdpTransceiverP had to be extenden to reflect
such requests as an ICMP port unreachable message.

Information on the ICMP protocol was taken from: https://en.wikipedia.org/wiki/Internet\_Control\_Message\_Protocol

%---------------------------------------------------------------------------
\subsection{PS2}
%---------------------------------------------------------------------------

In order to support PS2 communication, an ISR has to be installed on the PS2
clock line. Every time a falling edge is detected, the data line is sampled
into a shift register. After the reception of all PS2 bits, the data byte is 
decoded with the help of scancode tables for upper and lowercase. In order
to achieve maximum performance, the table lookup relies on binary search.
Upper-/lowercase tables are switched according to shift key states. After
the succesful decoding of the scancode, the obtained character is signaled
to the application.

%***************************************************************************
\section{FMCLick}
%***************************************************************************

This module handles the control of the FMClick radio module. Communication
happens over the I2C bus. The FMClick module contains 16 configuration registers,
which have to be read and written sequentially over the bus. In order to minimize
communication, the module uses a local shadow register, which buffers the current
register state of the FMClick module. In order to write to the radio module,
the registers are modified in the shadow register, which is then copied to a 
communication buffer in the right order (starting with the first write address).
This buffer is then written sequentially to the bus, up until the highest write
address. To perform a read operation, all registers are read sequentially from the
bus into the buffer and then written to the communication buffer in the right
order. If a driver command is called, the driver gets blocked until the command
is completed, as the commands require many different steps which must not be
interrupted. Init, seek and tune are implemented according to the state machines
provided in the application note. The seek interface was extendedwit a seekmode\_t
type, where it is possible to specify and additional BAND mode. This mode configures
the radio in such way, that it does not wrap around band limits. Reception of
RDS information can be en- and disbled with the respective command. If enabled, 
the RDS interrupt routine posts a decode RDS task, which reads the RDS information
from the radio module. These registers are then decoded according to the standard
(http://www.interactive-radio-system.com/docs/EN50067\_RDS\_Standard.pdf). The station name
and picode are only signalled to the application, when all blocks have been receive, while
the radio text is signalled everytime a block has been received, in order to limit
waiting time until information can be displayed. The time is signalled right away,
as it only requires one RDS message. To keep things simple, the half-hour timezone
offset is not considered. Reception of RDS information is disabled by seek and 
tune commands in order to avoid wrong interrupts, but reenabled afterwards if
it was enabled before.

%***************************************************************************
\section{Database}
%***************************************************************************

This module handles communication with the database server on the desktop PC.
To keep things simple, there are no message queues, thus after the sending of
one command, the module blocks until it has received and decoded a response
message. Therefore the module only needs to buffer one sended and one received
message, leading to a small memory footprint. The udpi\_msg\_t type in udp\_config.h
was extended with a len field, for more convenient message handling. If a module
command is called, the appropriate UDP message string is composed according to
the specified protocol from progmem and input. A send task is posted, which 
hands the message over to the UDP module. Afterwards the module waits for response,
and if we have onie, the message is again buffered and a decode task is started.
The message is decoded based on which command we expect a response to. Add and update
responses are simple to decode, we just have to check if the command has been 
executed successfully and then signal the result to the application. A list response
requires that all given IDs are stored. Afterwards a new fetchList task is posted,
which issues a getChannel request for every ID in the list. After the last database
item was received, 0xff is signalled to the application. Get responses require
the more work, as they provide the most information. The decoder iterates over
the parameter list, decodes and checks every item accordingly and puts it into
a channelInfo struct, which is then signalled to the application.

%***************************************************************************
\section{VolumeAdc}
%***************************************************************************

This module provides access to the ATMEGA1280 ADC in order to read input from
the potentiometer. TinyOS provides the necessary driver modules, so it just
required to configure the ADC with the desired settings and make configure
the potentiometer pin as an input.

%***************************************************************************
\section{Problems}
%***************************************************************************

The main problem was the handling of the FMClick module. It 
requires a lot of effort to understand exactly how it works and how it needs
to be configured in order to work in a desired manner. Also in the beginning
it was very hard to debug the driver, as the module seemed to not work properly.
This led to confusion whether the problems came from the software or the hardware.
After finding out that the module is very sensitive and requires stable connections,
this got a bit easier. In the rest of the application no major problems were
encountered.

%***************************************************************************
\section{Work}
%***************************************************************************

Estimate the work you put into solving the Application.

\begin{tabular}{|l|c|c|}\hline
	Task & Time spent \\ \hline

	reading manuals, datasheets &  5 h\\
	program design              &  2 h\\
	programming                 & 30 h\\
	debugging                   & 45 h\\
	questions, protocol         &  2 h\\ \hline

	\textbf{Total}              & 75 h\\ \hline
\end{tabular}

%***************************************************************************
\section{Theory Tasks}
%***************************************************************************

In the theory task we want you to develop argumentation skills that
     allow you to reason about the problem you have to solve and the
     solution you are designing.
Clear presentation of ideas is crucial for communication with team
     members, bosses, customers, etc.
This time we want you to prove your answers mathematically (by
     induction).
Points are solely awarded for proper mathematical argumentation.\\
\\
%%%%%%%%%%%%%%%%% new tasks %%%%%%%%%%%%%%%%%%%%%%%%%%%%%%%555
Setting:

A set of $n$ motes $\{p_1,\cdots,p_n\}$ is scattered in an area to broadcast data.
For simplicity, assume that every mote $p_i$ starts with a unique data sample $i$.
That is, mote $p_1$ starts with $1$, mote $p_2$ with $2$, and so on.

Communication is divided into several subsequent rounds and happens wirelessly via broadcasts.
Starting with round $r=1$, in each round every mote $i$ sends a broadcast message $m_i(r)$ to the system.
Broadcasting happens in zero time and all motes receive messages just before the next round starts.
A message $m_i(r)$ sent in round $r$ is the set of known data samples $S_i$. 
For example the message $m_1(1)$ sent by mote $1$ in round $1$ is $S_1=\{1\}$.
After receiving messages in some round $r$ mote $i$ updates the set of known data samples. 
E.g., if mote $p_i$ receives a message $m_j(r)$ from mote $p_j$ $S_i=S_i\cup S_j$.
This update is done for all received messages in each round.

This round-based communication can be modelled with a communication graph $G_r$
where each node in the graph represents a mote. 
A directed edge $(i,j)$ in $G_r$ represents that the message sent by mote $p_i$ in round $r$ was sucessfully received by mote $p_j$ in round $r$. 

\subsection*{Tasks} 

\begin{enumerate} 
	\item \pts{3 Points}{Strongly connected communication:}
		Assume that, because of message loss, not every message sent by a mote
		is received by every other mote. 
		The only guarantee the motes have is that in every round the
		communication graph is strongly connected.
		Show that after $n$ rounds of communication that data sample $1$ is part of every set $s_i$, i.e., $1 \in \bigcap_{i=1}^n S_i $.
\begin{figure}[h!]
\centering
	\begin{tikzpicture}
		\draw (0, 0.5) node (p1)    {$p_1$};
		\draw (-1, -1) node (p2)    {$p_2$};
		\draw (1, -1) node (p3)    {$p_3$};
		\draw[->] (p1) -- (p2);
		\draw[->] (p2) -- (p3);
		\draw[->] (p3) -- (p1);

		\draw (4, 0.5) node (p1)    {$p_1$};
		\draw (3, -1) node (p2)    {$p_2$};
		\draw (5, -1) node (p3)    {$p_3$};
		\draw[<->] (p1) -- (p2);
		\draw[<->] (p2) -- (p3);
		\draw[<->] (p3) -- (p1);

		\draw (8, 0.5) node (p1)    {$p_1$};
		\draw (7, -1) node (p2)    {$p_2$};
		\draw (9, -1) node (p3)    {$p_3$};
		\draw[<->] (p1) -- (p2);
		\draw[<->] (p2) -- (p3);	
	\end{tikzpicture}
	\caption{Example for $n=3$: three strongly connected graphs}
\end{figure}
	%\item \pts{1 Points}{Strongly connected communication extended:} 
	%	Under the same conditions as befor,
	%	how many rounds does it take for all motes to send the local unique data sample to all other motes \textbf(at most) (i.e. show that it is possible for $x$ rounds and not for $x-1$).
	\item \pts{2 Points}{Rooted tree communication:} Assume that,
		because of message loss, not every message send by a mote is received
		by every other mote. The only guarantee the motes have is that in every round
		the communication graph is a rooted tree, i.e., there exists at least one mote (the so called root mote) that has a directed path (a sequence of edges) to every other mote in $G_r$. Note that 		the root mote can be a different mote in every round.
		Show that after $n^2$ rounds of communication there exists a data sample $i$ that is part of every set $S_j$, i.e., $\{i\} \in \bigcap_{j=1}^n S_j $.
\begin{figure}[h!]
\centering
	\begin{tikzpicture}
		\draw (0, 0.5) node (p1)    {\textcolor{red}{$p_1$}};
		\draw (-1, -1) node (p2)    {$p_2$};
		\draw (1, -1)  node (p3)    {$p_3$};
		\draw[->] (p1) -- (p2);
		\draw[->] (p2) -- (p3);


		\draw (4, 0.5) node (p1)    {$p_1$};
		\draw (3, -1) node (p2)    {\textcolor{red}{$p_2$}};
		\draw (5, -1) node (p3)    {$p_3$};
		\draw[->] (p2) -- (p3);
		\draw[->] (p2) -- (p1);

	\end{tikzpicture}
	\caption{Example for $n=3$: two rooted trees}
\end{figure}
\end{enumerate}



% Your answers should be brief but complete




%***************************************************************************
\newpage
\end{document}

