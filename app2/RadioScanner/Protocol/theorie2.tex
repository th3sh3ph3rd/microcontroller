In the theory task we want you to develop argumentation skills that
     allow you to reason about the problem you have to solve and the
     solution you are designing.
Clear presentation of ideas is crucial for communication with team
     members, bosses, customers, etc.
This time we want you to prove your answers mathematically (by
     induction).
Points are solely awarded for proper mathematical argumentation.\\
\\
%%%%%%%%%%%%%%%%% new tasks %%%%%%%%%%%%%%%%%%%%%%%%%%%%%%%555
Setting:

A set of $n$ motes $\{p_1,\cdots,p_n\}$ is scattered in an area to broadcast data.
For simplicity, assume that every mote $p_i$ starts with a unique data sample $i$.
That is, mote $p_1$ starts with $1$, mote $p_2$ with $2$, and so on.

Communication is divided into several subsequent rounds and happens wirelessly via broadcasts.
Starting with round $r=1$, in each round every mote $i$ sends a broadcast message $m_i(r)$ to the system.
Broadcasting happens in zero time and all motes receive messages just before the next round starts.
A message $m_i(r)$ sent in round $r$ is the set of known data samples $S_i$. 
For example the message $m_1(1)$ sent by mote $1$ in round $1$ is $S_1=\{1\}$.
After receiving messages in some round $r$ mote $i$ updates the set of known data samples. 
E.g., if mote $p_i$ receives a message $m_j(r)$ from mote $p_j$ $S_i=S_i\cup S_j$.
This update is done for all received messages in each round.

This round-based communication can be modelled with a communication graph $G_r$
where each node in the graph represents a mote. 
A directed edge $(i,j)$ in $G_r$ represents that the message sent by mote $p_i$ in round $r$ was sucessfully received by mote $p_j$ in round $r$. 

\subsection*{Tasks} 

\begin{enumerate} 
	\item \pts{3 Points}{Strongly connected communication:}
		Assume that, because of message loss, not every message sent by a mote
		is received by every other mote. 
		The only guarantee the motes have is that in every round the
		communication graph is strongly connected.
		Show that after $n$ rounds of communication that data sample $1$ is part of every set $s_i$, i.e., $1 \in \bigcap_{i=1}^n S_i $.
\begin{figure}[h!]
\centering
	\begin{tikzpicture}
		\draw (0, 0.5) node (p1)    {$p_1$};
		\draw (-1, -1) node (p2)    {$p_2$};
		\draw (1, -1) node (p3)    {$p_3$};
		\draw[->] (p1) -- (p2);
		\draw[->] (p2) -- (p3);
		\draw[->] (p3) -- (p1);

		\draw (4, 0.5) node (p1)    {$p_1$};
		\draw (3, -1) node (p2)    {$p_2$};
		\draw (5, -1) node (p3)    {$p_3$};
		\draw[<->] (p1) -- (p2);
		\draw[<->] (p2) -- (p3);
		\draw[<->] (p3) -- (p1);

		\draw (8, 0.5) node (p1)    {$p_1$};
		\draw (7, -1) node (p2)    {$p_2$};
		\draw (9, -1) node (p3)    {$p_3$};
		\draw[<->] (p1) -- (p2);
		\draw[<->] (p2) -- (p3);	
	\end{tikzpicture}
	\caption{Example for $n=3$: three strongly connected graphs}
\end{figure}
	%\item \pts{1 Points}{Strongly connected communication extended:} 
	%	Under the same conditions as befor,
	%	how many rounds does it take for all motes to send the local unique data sample to all other motes \textbf(at most) (i.e. show that it is possible for $x$ rounds and not for $x-1$).
	\item \pts{2 Points}{Rooted tree communication:} Assume that,
		because of message loss, not every message send by a mote is received
		by every other mote. The only guarantee the motes have is that in every round
		the communication graph is a rooted tree, i.e., there exists at least one mote (the so called root mote) that has a directed path (a sequence of edges) to every other mote in $G_r$. Note that 		the root mote can be a different mote in every round.
		Show that after $n^2$ rounds of communication there exists a data sample $i$ that is part of every set $S_j$, i.e., $\{i\} \in \bigcap_{j=1}^n S_j $.
\begin{figure}[h!]
\centering
	\begin{tikzpicture}
		\draw (0, 0.5) node (p1)    {\textcolor{red}{$p_1$}};
		\draw (-1, -1) node (p2)    {$p_2$};
		\draw (1, -1)  node (p3)    {$p_3$};
		\draw[->] (p1) -- (p2);
		\draw[->] (p2) -- (p3);


		\draw (4, 0.5) node (p1)    {$p_1$};
		\draw (3, -1) node (p2)    {\textcolor{red}{$p_2$}};
		\draw (5, -1) node (p3)    {$p_3$};
		\draw[->] (p2) -- (p3);
		\draw[->] (p2) -- (p1);

	\end{tikzpicture}
	\caption{Example for $n=3$: two rooted trees}
\end{figure}
\end{enumerate}

